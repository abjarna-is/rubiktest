%% Generated by Sphinx.
\def\sphinxdocclass{report}
\documentclass[letterpaper,10pt,english]{sphinxmanual}
\ifdefined\pdfpxdimen
   \let\sphinxpxdimen\pdfpxdimen\else\newdimen\sphinxpxdimen
\fi \sphinxpxdimen=.75bp\relax
\ifdefined\pdfimageresolution
    \pdfimageresolution= \numexpr \dimexpr1in\relax/\sphinxpxdimen\relax
\fi
%% let collapsible pdf bookmarks panel have high depth per default
\PassOptionsToPackage{bookmarksdepth=5}{hyperref}

\PassOptionsToPackage{booktabs}{sphinx}
\PassOptionsToPackage{colorrows}{sphinx}

\PassOptionsToPackage{warn}{textcomp}
\usepackage[utf8]{inputenc}
\ifdefined\DeclareUnicodeCharacter
% support both utf8 and utf8x syntaxes
  \ifdefined\DeclareUnicodeCharacterAsOptional
    \def\sphinxDUC#1{\DeclareUnicodeCharacter{"#1}}
  \else
    \let\sphinxDUC\DeclareUnicodeCharacter
  \fi
  \sphinxDUC{00A0}{\nobreakspace}
  \sphinxDUC{2500}{\sphinxunichar{2500}}
  \sphinxDUC{2502}{\sphinxunichar{2502}}
  \sphinxDUC{2514}{\sphinxunichar{2514}}
  \sphinxDUC{251C}{\sphinxunichar{251C}}
  \sphinxDUC{2572}{\textbackslash}
\fi
\usepackage{cmap}
\usepackage[T1]{fontenc}
\usepackage{amsmath,amssymb,amstext}
\usepackage{babel}



\usepackage{tgtermes}
\usepackage{tgheros}
\renewcommand{\ttdefault}{txtt}



\usepackage[Bjarne]{fncychap}
\usepackage{sphinx}

\fvset{fontsize=auto}
\usepackage{geometry}

\usepackage{sphinxcontribtikz}

% Include hyperref last.
\usepackage{hyperref}
% Fix anchor placement for figures with captions.
\usepackage{hypcap}% it must be loaded after hyperref.
% Set up styles of URL: it should be placed after hyperref.
\urlstyle{same}

\addto\captionsenglish{\renewcommand{\contentsname}{Contents:}}

\usepackage{sphinxmessages}
\setcounter{tocdepth}{1}



\title{rubiktest\sphinxhyphen{}projectname}
\date{May 12, 2024}
\release{0.1.0}
\author{rubiktest\sphinxhyphen{}authorname}
\newcommand{\sphinxlogo}{\vbox{}}
\renewcommand{\releasename}{Release}
\makeindex
\begin{document}

\ifdefined\shorthandoff
  \ifnum\catcode`\=\string=\active\shorthandoff{=}\fi
  \ifnum\catcode`\"=\active\shorthandoff{"}\fi
\fi

\pagestyle{empty}
\sphinxmaketitle
\pagestyle{plain}
\sphinxtableofcontents
\pagestyle{normal}
\phantomsection\label{\detokenize{index::doc}}


\sphinxAtStartPar
Smá meira testing test.

\sphinxAtStartPar
Since Pythagoras, we know that \(a^2 + b^2 = c^2\).
\begin{equation}\label{equation:index:euler}
\begin{split}e^{i\pi} + 1 = 0\end{split}
\end{equation}
\sphinxAtStartPar
Euler’s identity, equation \eqref{equation:index:euler}, was elected one of the most
beautiful mathematical formulas.

\sphinxAtStartPar
\(\underline{x}=[  x_{1}, ...,  x_{n}]^{T}\)
\begin{quote}

\sphinxAtStartPar
Setjum \(b_y=-6b_x\) inn og fáum:
\end{quote}
\begin{equation*}
\begin{split}9 = \sqrt{b_x^2+b_y^2}\end{split}
\end{equation*}\begin{equation*}
\begin{split}\begin{aligned}
9 &= \sqrt{b_x^2+b_y^2}\\
&=\sqrt{b_x^2+(-6b_x)^2} \\
&= \sqrt{b_x^2+36b_x^2} \\
&=\sqrt{37b_x^2} \\
&=b_x\sqrt{37} \\
b_x&=\frac{9}{\sqrt{37}} \approx 1.480\\
b_y&= -6b_x = \frac{-54}{\sqrt{37}} \approx -8.878
\end{aligned}\end{split}
\end{equation*}
\sphinxAtStartPar
Vigur sem er samsíða \(\overline{a}=(-1,6)\) og hefur lengdina 9 er því
\begin{equation*}
\begin{split}\overline{b}= \begin{pmatrix} \frac{9}{\sqrt{37}} \\  \frac{-54}{\sqrt{37}} \end{pmatrix}\end{split}
\end{equation*}
\begin{sphinxadmonition}{note}{Dæmi og lausn}

\sphinxAtStartPar
Hér er dæmi og lausn
\end{sphinxadmonition}

\begin{sphinxadmonition}{note}{Annað dæmi og lausn sem er hægt að opna og loka}

\sphinxAtStartPar
Hér er annað dæmi og lausn
\end{sphinxadmonition}

\begin{sphinxadmonition}{note}{Dæmi og lausn}
\begin{figure}[H]\raggedright\capstart\begin{tikzpicture}[domain=0:5,scale=1,thick]
\usetikzlibrary{calc}                                %allows coordinate calculations.
\usetikzlibrary{decorations.pathreplacing}           %allows drawing curly braces.
\def\dint{4.5}      %Y-intercept for DEMAND.
\def\dslp{-0.5}     %Slope for DEMAND.
\def\sint{1.2}      %Y-intercept for SUPPLY.
\def\sslp{0.8}      %Slope for SUPPLY.
\def\tax{1.5}       %Excise (per-unit) tax
  \def\demand{\x,{\dslp*\x+\dint}}
  \def\supply{\x,{\sslp*\x+\sint}}
  \def\demandtwo{\x,{\dslp*\x+\dint+\dsh}}
  \def\supplytwo{\x,{\sslp*\x+\sint+\ssh}}
    \coordinate (ints) at ({(\sint-\dint)/(\dslp-\sslp)},{(\sint-\dint)/(\dslp-\sslp)*\sslp+\sint});
    \coordinate (ep) at  (0,{(\sint-\dint)/(\dslp-\sslp)*\sslp+\sint});
    \coordinate (eq) at  ({(\sint-\dint)/(\dslp-\sslp)},0);
    \coordinate (dint) at (0,{\dint});
    \coordinate (sint) at (0,{\sint});
    \coordinate (teq) at  ({(\sint+\tax-\dint)/(\dslp-\sslp)},0); %quantity
    \coordinate (tep) at  (0,{(\sint+\tax-\dint)/(\dslp-\sslp)*\sslp+\sint+\tax}); %price
    \coordinate (tint) at  ({(\sint+\tax-\dint)/(\dslp-\sslp)},{(\sint+\tax-\dint)/(\dslp-\sslp)*\sslp+\sint+\tax}); %tax equilibrium
    \coordinate (sep) at (0,{\sslp*(\sint+\tax-\dint)/(\dslp-\sslp)+\sint});
    \coordinate (sen) at ({(\sint+\tax-\dint)/(\dslp-\sslp)},{\sslp*(\sint+\tax-\dint)/(\dslp-\sslp)+\sint});   
    \draw[thick,color=blue] plot (\demand) node[right] {$P(q) = -\frac{1}{2}q+\frac{9}{2}$};
    \draw[thick,color=purple] plot (\supply) node[right] {Supply};
    \draw[->] (0,0) -- (6.2,0) node[right] {$Q$};
    \draw[->] (0,0) -- (0,6.2) node[above] {$P$};
    \draw[decorate,decoration={brace},thick]  ($(sep)+(-0.8,0)$) -- ($(tep)+(-0.8,0)$) node[midway,below=-8pt,xshift=-18pt] {tax};
    \draw[dashed] (tint) -- (teq) node[below] {$Q_T$};                  
    \draw[dashed] (tint) -- (tep) node[left] {$P_d$};                       
    \draw[dashed] (sen) -- (sep) node[left] {$P_s$};                        
\end{tikzpicture}
\caption{test2.tex \textendash{} lkj lkj lkj lkj lkj lkj lkjlkjlkjlkj lkj lkj lkj lk jlkj lkj lk jlkj lkj lkJ LKJ}\label{\detokenize{index:id1}}\end{figure}\end{sphinxadmonition}

\sphinxAtStartPar
test.tex
\begin{flushleft}\begin{tikzpicture}[domain=0:5,scale=1,thick]
\usetikzlibrary{calc}                                %allows coordinate calculations.
\usetikzlibrary{decorations.pathreplacing}           %allows drawing curly braces.
\def\dint{4.5}      %Y-intercept for DEMAND.
\def\dslp{-0.5}     %Slope for DEMAND.
\def\sint{1.2}      %Y-intercept for SUPPLY.
\def\sslp{0.8}      %Slope for SUPPLY.
\def\tax{1.5}       %Excise (per-unit) tax
  \def\demand{\x,{\dslp*\x+\dint}}
  \def\supply{\x,{\sslp*\x+\sint}}
  \def\demandtwo{\x,{\dslp*\x+\dint+\dsh}}
  \def\supplytwo{\x,{\sslp*\x+\sint+\ssh}}
    \coordinate (ints) at ({(\sint-\dint)/(\dslp-\sslp)},{(\sint-\dint)/(\dslp-\sslp)*\sslp+\sint});
    \coordinate (ep) at  (0,{(\sint-\dint)/(\dslp-\sslp)*\sslp+\sint});
    \coordinate (eq) at  ({(\sint-\dint)/(\dslp-\sslp)},0);
    \coordinate (dint) at (0,{\dint});
    \coordinate (sint) at (0,{\sint});
    \coordinate (teq) at  ({(\sint+\tax-\dint)/(\dslp-\sslp)},0); %quantity
    \coordinate (tep) at  (0,{(\sint+\tax-\dint)/(\dslp-\sslp)*\sslp+\sint+\tax}); %price
    \coordinate (tint) at  ({(\sint+\tax-\dint)/(\dslp-\sslp)},{(\sint+\tax-\dint)/(\dslp-\sslp)*\sslp+\sint+\tax}); %tax equilibrium
    \coordinate (sep) at (0,{\sslp*(\sint+\tax-\dint)/(\dslp-\sslp)+\sint});
    \coordinate (sen) at ({(\sint+\tax-\dint)/(\dslp-\sslp)},{\sslp*(\sint+\tax-\dint)/(\dslp-\sslp)+\sint});   
    \draw[thick,color=blue] plot (\demand) node[right] {$P(q) = -\frac{1}{2}q+\frac{9}{2}$};
    \draw[thick,color=purple] plot (\supply) node[right] {Supply};
    \draw[->] (0,0) -- (6.2,0) node[right] {$Q$};
    \draw[->] (0,0) -- (0,6.2) node[above] {$P$};
    \draw[decorate,decoration={brace},thick]  ($(sep)+(-0.8,0)$) -- ($(tep)+(-0.8,0)$) node[midway,below=-8pt,xshift=-18pt] {tax};
    \draw[dashed] (tint) -- (teq) node[below] {$Q_T$};                  
    \draw[dashed] (tint) -- (tep) node[left] {$P_d$};                       
    \draw[dashed] (sen) -- (sep) node[left] {$P_s$};                        
\end{tikzpicture}
\end{flushleft}
\sphinxAtStartPar
Verðgólf inn í RST
\begin{center}\begin{tikzpicture}[domain=0:5,scale=1,thick]
\usetikzlibrary{calc}   %allows coordinate calculations.
\def\dint{4.5}          %Y-intercept for DEMAND.
\def\dslp{-0.5}         %Slope for DEMAND.
\def\sint{1.2}          %Y-intercept for SUPPLY.
\def\sslp{0.8}          %Slope for SUPPLY.
\def\pfc{2.5}           %Price floor or ceiling
\def\demand{\x,{\dslp*\x+\dint}}
\def\supply{\x,{\sslp*\x+\sint}}
   \coordinate (ints) at ({(\sint-\dint)/(\dslp-\sslp)},{(\sint-\dint)/(\dslp-\sslp)*\sslp+\sint});
   \coordinate (ep) at  (0,{(\sint-\dint)/(\dslp-\sslp)*\sslp+\sint});
   \coordinate (eq) at  ({(\sint-\dint)/(\dslp-\sslp)},0);
   \coordinate (dint) at (0,{\dint});
   \coordinate (sint) at (0,{\sint});
   \coordinate (pfq) at  ({(\pfc-\dint)/(\dslp)},0);
   \coordinate (pfp) at  ({(\pfc-\dint)/(\dslp)},{\pfc});
   \coordinate (sfq) at  ({(\pfc-\sint)/(\sslp)},0);
   \coordinate (sfp) at  ({(\pfc-\sint)/(\sslp)},{\pfc});
   \draw[thick,color=blue] plot (\demand) node[right] {$P(q) = -\frac{1}{2}q+\frac{9}{2}$};
   \draw[thick,color=purple] plot (\supply) node[right] {Supply};
   \draw[->] (0,0) -- (6.2,0) node[right] {$Q$};
   \draw[->] (0,0) -- (0,6.2) node[above] {$P$};
   \draw[dashed,color=black] plot (\x,{\pfc}) node[right] {$P_c$};
   \draw[dashed] (pfp) -- (pfq) node[below] {$Q_d$};
   \draw[dashed] (sfp) -- (sfq) node[below] {$Q_s$};
\draw[->,baseline=5] ($(0,{\pfc})+(-1.5,0.7)$) node[label= left:Price Ceiling] {} -- ($(0,{\pfc})+(-.1,0.1)$);
\end{tikzpicture}\end{center}
\sphinxAtStartPar
test2.tex
\begin{flushleft}\begin{tikzpicture}
\begin{axis}[
scale = 1.2,
xmin = 0, xmax = 10,
ymin = 0, ymax = 10,
axis lines* = left,
xtick = {0}, ytick = \empty,
clip = false,
]
\addplot[domain = 0:10, restrict y to domain = 0:10, samples =
   400, color = red]{10/(x^2)+1};
\addplot[domain = 1:10, restrict y to domain = 0:10, samples =
   400, color = red]{10/((x-1)^2)+2};
\addplot[domain = 2:10, restrict y to domain = 0:10, samples =
   400, color = red]{10/((x-2)^2)+3};
\addplot[domain = 3:10, restrict y to domain = 0:10, samples =
   400, color = red]{10/((x-3)^2)+4};
\addplot[domain = 4:10, restrict y to domain = 0:10, samples =
   400, color = red]{10/((x-4)^2)+5};
\addplot[domain = 0:10, restrict y to domain = 0:10, samples =
   400, color = blue, thick]{9.16-1.02*x};
\addplot[domain = 0:10, restrict y to domain = 0:10, samples =
   400, color = blue, thick]{9.16-1.59*x};
\addplot[color = black, dashed, thick] coordinates {(4.7, 0) (4.7,
    4.37) (0, 4.37)};
\addplot[color = black, dashed, thick] coordinates {(3.3, 0) (3.3,
    3.9) (0, 3.9)};
\addplot[color = black, mark = *, only marks, mark size = 3pt]
   coordinates {(3.3, 3.9) (4.7, 4.37)};
\node [right] at (current axis.right of origin) {$A$};
\node [above] at (current axis.above origin) {$B$};
\node [above] at (9.1, 0) {$M$};
\node [above] at (6, 0) {$M^\prime$};
\node [below] at (4.7, 0) {$Q_A$};
\node [below] at (3.3, 0) {$Q_A^\prime$};
\node [left] at (0, 4.5) {$Q_B$};
\node [left] at (0, 3.7) {$Q_B^\prime$};
\node [above] at (5.1, 4.2) {$D$};
\node [above] at (2.9, 2.8) {$D^\prime$};
\node [right] at (10, 1.1) {$U_1$};
\node [right] at (10, 2.12) {$U_2$};
\node [right] at (10, 3.16) {$U_3$};
\node [right] at (10, 4.2) {$U_4$};
\node [right] at (10, 5.4) {$U_5$};
\end{axis}
\end{tikzpicture}
\end{flushleft}
\sphinxAtStartPar
test3.tex
\begin{flushleft}\begin{tikzpicture}
\begin{axis}[
scale = 1.2,
xmin = 0, xmax = 10,
ymin = 0, ymax = 10,
axis lines* = left,
xtick = {0}, ytick = \empty,
axis on top,
clip = false,
]
\fill[orange, opacity = 0.1] (0, 7.67) -- (0, 3.67) -- (5.5, 3.67)
    -- (5.5, 7.67);
\fill[green, opacity = 0.1] (5.5, 0) -- (5.5, 3.67) -- (7, 3.67)
-- (7, 0);
\addplot[color = blue, very thick] coordinates {(5,9) (8,1)};
\addplot[color = red, very thick] coordinates {(3,1) (6,9)};
\addplot[color = red, opacity = 0.3, very thick] coordinates
   {(6,1) (9,9)};
\addplot[color = black, dashed, thick] coordinates {(0, 7.67)
   (5.5, 7.67) (5.5, 0)};
\addplot[color = black, dashed, thick] coordinates {(0, 3.67) (7,
   3.67) (7, 0)};
\addplot[color = black, mark = *, only marks, mark size = 3pt]
   coordinates {(5.5, 7.67) (7, 3.67)};
\node [right] at (current axis.right of origin){Farm products, $Q
   $};
\node [above] at (current axis.above origin) {Price, $P$};
\node [right] at (5.5, 7.67) {$E$};
\node [right] at (7, 3.67) {$E^\prime$};
\node [right] at (8, 1) {$D$};
\node [above] at (6, 9) {$S$};
\node [right] at (9, 9) {$S^\prime$};
\node [left] at (0, 7.67) {$P_E$};
\node [left] at (0, 3.67) {$P^\prime$};
\node [below] at (5.5, 0) {$Q_E$};
\node [below] at (7, 0) {$Q^\prime$};
\node [right] at (5, 11) {Total revenue lost};
\node [above] at (10, -3) {Total revenue gained};
\draw[-{Triangle[length = 4mm, width = 2mm]}, red, opacity = 0.3]
   (6.3, 8.5) to (8.3, 8.5);
\draw[-Triangle] (5, 11) to [out = 180, in = 90] (2, 6);
\draw[-Triangle] (10, -1.5)  to [out = 90, in = 0] (6.5, 0.7);
\end{axis}
\end{tikzpicture}
\end{flushleft}
\sphinxAtStartPar
test\_table.tex
\begin{flushleft}\begin{tikzpicture}
\node {\scalebox{2}{% Scale by factor of 2
\pgfplotstabletypeset
    [col sep=&,row sep=\\,sci zerofill]
{
    level &  dof &    error \\
    1 &      4 &      2.50000000e-01 \\
    2 &      16 &     6.25000000e-02 \\
    3 &      64 &     1.56250000e-02 \\
    4 &      256 &    3.90625000e-03 \\
    5 &      1024 &   9.76562500e-04 \\
    6 &      4096 &   2.44140625e-04 \\
    7 &      16384 &  6.10351562e-05 \\
    8 &      65536 &  1.52587891e-05 \\
    9 &      262144 & 3.81469727e-06 \\
    10 &     1048576 &9.53674316e-07 \\
}
   }
};
\end{tikzpicture}
\end{flushleft}
\sphinxAtStartPar
test\_table2.tex
\begin{flushleft}\begin{tikzpicture}
\node {\scalebox{1}{
\pgfplotstabletypeset[
    col sep=&,row sep=\\,sci zerofill,
    every head row/.style={
        before row=\toprule,after row=\midrule},
    every last row/.style={
        after row=\bottomrule},
    columns/level/.style={column type={>{\centering\arraybackslash}p{.3\linewidth}}},
    columns/dof/.style={column type={>{\centering\arraybackslash}p{.3\linewidth}}},
    columns/error/.style={column type={>{\centering\arraybackslash}p{.3\linewidth}}},
]
{
    level &  dof &    error \\
    1 &      4 &      2.50000000e-01 \\
    2 &      16 &     6.25000000e-02 \\
    3 &      64 &     1.56250000e-02 \\
    4 &      256 &    3.90625000e-03 \\
    5 &      1024 &   9.76562500e-04 \\
    6 &      4096 &   2.44140625e-04 \\
    7 &      16384 &  6.10351562e-05 \\
    8 &      65536 &  1.52587891e-05 \\
    9 &      262144 & 3.81469727e-06 \\
    10 &     1048576 &9.53674316e-07 \\
}
   }
};
\end{tikzpicture}
\end{flushleft}

\begin{savenotes}\sphinxattablestart
\sphinxthistablewithglobalstyle
\centering
\begin{tabulary}{\linewidth}[t]{TT}
\sphinxtoprule
\sphinxstyletheadfamily 
\sphinxAtStartPar
Header 1
&\sphinxstyletheadfamily 
\sphinxAtStartPar
Header 2
\\
\sphinxmidrule
\sphinxtableatstartofbodyhook
\sphinxAtStartPar
1
&
\sphinxAtStartPar
one
\\
\sphinxhline
\sphinxAtStartPar
1,5
&
\sphinxAtStartPar
test
\\
\sphinxhline
\sphinxAtStartPar
2
&
\sphinxAtStartPar
two
\\
\sphinxbottomrule
\end{tabulary}
\sphinxtableafterendhook\par
\sphinxattableend\end{savenotes}


\chapter{Indices and tables}
\label{\detokenize{index:indices-and-tables}}\begin{itemize}
\item {} 
\sphinxAtStartPar
\DUrole{xref,std,std-ref}{genindex}

\item {} 
\sphinxAtStartPar
\DUrole{xref,std,std-ref}{modindex}

\item {} 
\sphinxAtStartPar
\DUrole{xref,std,std-ref}{search}

\end{itemize}



\renewcommand{\indexname}{Index}
\printindex
\end{document}